\documentclass{./RiaoOptilas2016}

\begin{document}

% TITLE OF THE CONTRIBUTION
\title{Nonlinear optical responses in hydrogenated graphene structures}

% LIST OF AUTHORS AND AFFILIATIONS 
\author{Reinaldo Zapata-Pe\~na$^{1}$, Sean M. Anderson$^{1}$, Bernardo S. 
Mendoza$^{1}$, Anatoli I. Shkrebtii$^{2}$ }
\bigskip

\address{
$^1$Centro de Investigaciones en \'Optica, Le\'on, Guanajuato 37150, M\'exico\\
\bigskip
$^2$University of Ontario, Institute of Technology, Oshawa, ON, L1H 7L7, Canada}

% EMAIL OF THE CORRESPONDING AUTHOR
\email{a.bxc@cio.mx, bms@cio.mx}

% ABSTRACT 
The text of the abstract is limited to a maximum of about 400 words \cite{Author82}. The page must be formatted in an A4 standard size (21 cm x 29.7 cm). For the left, right and top margins, please use 2 cm. All text must be written in 12 point Times New Roman font and in \textbf{English and Spanish or Portuguese}. The text of the abstract should be single-spaced and should be fully right and left justified \cite{Author96}. Use one separation line as indicated in this template. Figures can be included in the abstract. Underline the name of the author who will present the paper at the conference.

\bigskip

Abstracts must be sent by using the sending form or as attached files (.doc, .docx, .tex) to \\ \textbf{riao.optilas2016@cefop.udec.cl}

\keywords{Keyword1; keyword2; keyword3}

\ociscodes{(XXX.OOO1)}

% REFERENCES

\begin{thebibliography}{99}

\bibitem{Author82} Au. Thor1, Au. Thor2, Au.Thor3, Optics Express \textbf{35}, 2145-2152 (2013).

\bibitem{Author96} A. Thor1, A. Thor2, A.Thor3, Physical Review A \textbf{145}, 023456 (2014).

\end{thebibliography}

\end{document}

We present a theoretical study of the optical spin injection, optical current
injection, and second harmonic generation of two 50\% hydrogenated graphene
structures: C$_{16}$H$_{8}$-alt and C$_{16}$H$_{8}$-up. Optical spin injection,
under the incidence of circularly polarized light onto nonmagnetic
semiconductors, creates spin-polarized electrons in the conduction bands.
Optical current injection and second-harmonic generation are nonlinear second-
order effects that are allowed in materials without inversion symmetry. The
results are calculated in a full electronic band structure scheme within DFT in
the LDA approximation. Our results show an anisotropic behavior in the optical
responses of the spin, the current injection and second harmonic generation. We
obtained maximum absolute magnitudes of the degree of spin polarization of 61\%
and 64\% for the \emph{alt} and \emph{up} structures, respectively, and it is
also possible to optically generate an injection current coming mainly from the
carbon layer on both \emph{alt} and \emph{up} systems. Besides, we found that
both structures are excellent candidates for second harmonic generation.
