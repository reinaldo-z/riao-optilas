\documentclass{./RiaoOptilas2016}

\begin{document}

% TITLE OF THE CONTRIBUTION
\title{Nonlinear optical responses in hydrogenated graphene structures}

% LIST OF AUTHORS AND AFFILIATIONS 
\author{Reinaldo Zapata-Pe\~na$^{1}$, Sean M. Anderson$^{1}$, Bernardo S. 
Mendoza$^{1}$, Anatoli I. Shkrebtii$^{2}$ }
\bigskip

\address{
$^1$Centro de Investigaciones en \'Optica, Le\'on, Guanajuato 37150, M\'exico\\
\bigskip
$^2$University of Ontario, Institute of Technology, Oshawa, ON, L1H 7L7, Canada}

% EMAIL OF THE CORRESPONDING AUTHOR
\email{a.bxc@cio.mx, bms@cio.mx}

% ABSTRACT 
We present a theoretical study of the optical spin injection, optical current
injection, and second harmonic generation of two 50\% hydrogenated graphene
structures: C$_{16}$H$_{8}$-alt and C$_{16}$H$_{8}$-up. Optical spin
injection, under the incidence of circularly polarized light onto nonmagnetic
semiconductors, creates spin-polarized electrons in the conduction bands
\cite{dyakonovOO84}. Optical current injection\cite{arzatePRB14} and second-
harmonic generation\cite{loudonOUP00,andersonPRB15} are nonlinear second-
order effects that are allowed in materials without inversion symmetry. The
results are calculated in a full electronic band structure scheme within DFT
in the LDA approximation. Our results show an anisotropic behavior in the
optical responses of the spin injection, the current injection and second
harmonic generation. We obtained maximum absolute magnitudes of the degree of
spin polarization of 61\% and 64\% for the \emph{alt} and \emph{up}
structures, respectively, and it is also possible to optically generate an
injection current coming mainly from the carbon layer on both \emph{alt} and
\emph{up} systems. Besides, we found that both structures are excellent
candidates for second harmonic generation.

Presentamos un estudio te\'orico de injecci\'on \'optica de spin, inyecci\'on
\'optica de corriente y generaci\'on de segundo arm\'onico de dos estructuras
de grafeno hidrogenado al 50\%: C$_{16}$H$_{8}$-alt and C$_{16}$H$_{8}$-up.
La inyección \'optica de spin, bajo la incidencia de luz circularmenet
polarizada sobre semiconductores no magn\'eticos, crea electrones con spin
polarizado en las bandas de conducci\'on\cite{dyakonovOO84}. La inyecci\'on
\'optica de corriente\cite{arzatePRB14} y la generaci\'on de segundo
arm\'onico \cite{loudonOUP00,andersonPRB15} son efectos no lineales de
segundo orden que se presentan en materiales sin inversi\'on de simetr\'ia.
Los resultados fueron calculados usando un esquema de estructura de bandas
electrónica completa usando DFT con una aproximaci\'on LDA. Nuestros
resultados muestran un comportamiento anisotr\'opico en las respuestas
\'opticas de inyecci\'on de spin, inuyecci\'on de corriente y generaci\'on de
segundo arm\'onico. Obtuvimos magnituds absolutas m\'aximas del grado de
polarizaci\'on de spin de 61\% y 64\% para las estructuras \emph{alt} y
\emph{up}, respectivamente, y adem\'as es posible generar inyecci\'on de
corriente que proviene principalmente de la capa de carbonos de ambos
sistemas. Tambi\'en encontramos que ambas estructuras son excelentes
candidatos para la generaci\'on de segundo arm\'onico.


\bigskip

\keywords{graphene; spin polarization; current injection; second-harmonic.}

\ociscodes{(XXX.OOO1)}

% REFERENCES
\begin{thebibliography}{99}

\bibitem{dyakonovOO84} M. I. Dyakonov, V. I. Perel, Theory of optical spin
orientation of electrons and nuclei in semiconductors, Chapter 2, Elsevier,
NY, 1984.

\bibitem{arzatePRB14} N. Arzate, R.A. V{\'a}zquez-Nava, B.S. Mendoza, Physical Review B \textbf{90}(20), 205310 (2014).

\bibitem{loudonOUP00} R. Loudon, The Quantum Theory of Light, Oxford University Press, Oxford, 2000.

\bibitem{andersonPRB15} S.M. Anderson, N. Tancogne-Dejean, B.S. Mendoza, V. V{\'e}niard, Physical Review B \textbf{91}(7), 075302 (2015).

\end{thebibliography}

\end{document}

The text of the abstract is limited to a maximum of about 400 words \cite{Author82}. The page must be formatted in an A4 standard size (21 cm x 29.7 cm). For the left, right and top margins, please use 2 cm. All text must be written in 12 point Times New Roman font and in \textbf{English and Spanish or Portuguese}. The text of the abstract should be single-spaced and should be fully right and left justified \cite{Author96}. Use one separation line as indicated in this template. Figures can be included in the abstract. Underline the name of the author who will present the paper at the conference.

Abstracts must be sent by using the sending form or as attached files (.doc, .docx, .tex) to \\ \textbf{riao.optilas2016@cefop.udec.cl}






